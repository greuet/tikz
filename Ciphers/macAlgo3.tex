%% Author: Aurélien Greuet (aureliengreuet@gmail.com)
%%
%% This work is licensed under the Creative Commons Attribution
%% Attribution-ShareAlike 4.0 International (CC BY-SA 4.0).
%% To view a copy of this license, visit
%% https://creativecommons.org/licenses/by-sa/4.0/

\documentclass{article}
\usepackage[utf8]{inputenc}
\usepackage{tikz}
\usepackage[graphics,tightpage,active]{preview}


\usetikzlibrary{backgrounds,calc,shapes,shapes.symbols, fit,
  decorations.pathmorphing, decorations.pathreplacing, positioning}
\PreviewEnvironment{tikzpicture}


\definecolor{Blue}{RGB}{0,81,158}
\definecolor{Orange}{RGB}{233,93,45}
\definecolor{VioletBlue}{RGB}{87,35,129}
\definecolor{Red}{RGB}{226,0,43}
\definecolor{Green}{RGB}{0,134,51}


\tikzset{table/.style={matrix, column sep=-\pgflinewidth, nodes={rectangle}, 
    minimum width=0.8em, minimum height=0.8em,
    inner sep = 0}}
\tikzset{case/.style={draw, nodes={rectangle}, fill=Green!60,}}
\tikzset{case cipher/.style={draw, nodes={rectangle}, fill=Orange!60,}}
\tikzset{case IV/.style={draw, nodes={rectangle}, fill=VioletBlue!60,}}
\tikzset{BC/.style={draw, nodes={rectangle}, fill=Blue!60, text width=2.2cm, text centered}}
\tikzset{key/.style={draw, nodes={rectangle}, fill=Red!60, text centered}}
\tikzset{xor/.style={inner sep=0pt}}

\newcommand{\myarrow}{
  \begin{tikzpicture}
    \draw[->] (0,0.3) -- (0,0);
  \end{tikzpicture}
}



\begin{document}
\sffamily
\begin{tikzpicture}[node distance = 35pt,>=latex]
  %%--- middle block
  \matrix (bloc milieu plaintext)  [table% , below of = plaintext
  ]{
    \node[case] {}; & \node[case] {}; &  \node[case] {}; &  \node[case] {}; &  \node[case] {}; &  \node[case] {}; &  \node[case] {}; &  \node[case] {}; \\
  };
  \node[above = 0.5pt of bloc milieu plaintext] (bloc milieu plaintext label)
  {Plaintext $P_2$};
  \node (bloc milieu xor) [xor, below of = bloc milieu plaintext] {\LARGE $\oplus$};

  \node[BC, below = 20pt of bloc milieu xor] (bloc milieu BC) {Simple DES Encryption};
  \node[key, above = 10pt of bloc milieu BC.30] (bloc milieu key) {$K$};

  \draw[->, thick] (bloc milieu key) -- (bloc milieu BC.30);
  \draw[->, thick] (bloc milieu plaintext.south) -- (bloc milieu xor.north);
  \draw[->, thick] (bloc milieu xor.south) -- (bloc milieu BC.north);

  \node[fit=(bloc milieu BC)(bloc milieu plaintext)] (bloc milieu){};
  \node[fit=(bloc milieu BC)(bloc milieu plaintext)(bloc milieu key)] (bloc milieu full){};
  %%---


  %%--- left block
  \matrix (bloc gauche plaintext)  [table,left = 80pt of bloc milieu plaintext]{
    \node[case] {}; & \node[case] {}; &  \node[case] {}; &  \node[case] {}; &  \node[case] {}; &  \node[case] {}; &  \node[case] {}; &  \node[case] {}; \\
  };
  \node[above = 0.5pt of bloc gauche plaintext] (bloc gauche plaintext label)
  {Plaintext $P_1$};
  \node (bloc gauche xor) [xor, below of = bloc gauche plaintext] {\LARGE $\oplus$};
  \matrix (bloc gauche IV)  [table, left = 30 pt of bloc gauche xor]{
    \node[case IV] {}; & \node[case IV] {}; &  \node[case IV] {}; &  \node[case IV] {}; &  \node[case IV] {}; &  \node[case IV] {}; &  \node[case IV] {}; &  \node[case IV] {}; \\
  };
  \node (IV label) [above=1pt of bloc gauche IV] {\footnotesize Initialization Vector};  
  \node[BC, below = 20pt of bloc gauche xor] (bloc gauche BC) {Simple DES Encryption};
  \node[key, above=10 pt of bloc gauche BC.30] (bloc gauche key) {$K$};


  \draw[->, thick] (bloc gauche key) -- (bloc gauche BC.30);
  \draw[->, thick] (bloc gauche plaintext.south) -- (bloc gauche xor.north);
  \draw[->, thick] (bloc gauche xor.south) -- (bloc gauche BC.north);
  \draw[->, thick] (bloc gauche IV.east) -- (bloc gauche xor.west);


  \node[fit=(bloc gauche BC)(bloc gauche plaintext)] (bloc gauche){};
  \node[fit=(bloc gauche BC)(bloc gauche plaintext)(bloc gauche key)]
  (bloc gauche full){};
  %%---

  \draw[->, thick] (bloc gauche BC.south) -- ++ (0,-0.5) -- ++(2.5,0) 
  |- (bloc milieu xor.west) node[anchor = south east] {Ciphertext $C_1$};


  %%--- right block
  \matrix (bloc droit plaintext)  [table,right = 110pt of bloc milieu plaintext]{
    \node[case] {}; & \node[case] {}; &  \node[case] {}; &  \node[case] {}; &  \node[case] {}; &  \node[case] {}; &  \node[case] {}; &  \node[case] {}; \\
  };
  \node[above = 0.5pt of bloc droit plaintext] (bloc droit plaintext label)
  {Plaintext $P_n$};
  \node (bloc droit xor) [xor, below of = bloc droit plaintext] {\LARGE $\oplus$};
  \node[BC, below = 20pt of bloc droit xor] (bloc droit BC) {Triple DES Encryption};
  \node[key, above=10 pt of bloc droit BC.30] (bloc droit key) {$K, K'$};


  \draw[->, thick] (bloc droit key) -- (bloc droit BC.30);
  \draw[->, thick] (bloc droit plaintext.south) -- (bloc droit xor.north);
  \draw[->, thick] (bloc droit xor.south) -- (bloc droit BC.north);

  \matrix (bloc droit ciphertext)  [table, below=30pt of bloc droit BC]{
    \node[case cipher] {}; & \node[case cipher] {}; & \node[case cipher] {}; & \node[case cipher] {}; & \node[case cipher] {}; &
    \node[case cipher] {}; & \node[case cipher] {}; & \node[case cipher] {}; \\
  };
  \node[below = 0.5pt of bloc droit ciphertext] (bloc droit ciphertext label) {MAC};
  \draw[->, thick] (bloc droit BC.south) -- (bloc droit ciphertext.north);

  \node[fit=(bloc droit BC)(bloc droit ciphertext)(bloc droit plaintext)] (bloc droit){};
  \node[fit=(bloc droit BC)(bloc droit ciphertext)(bloc droit plaintext)(bloc droit key)] (bloc droit full){};
 %%---
 % save block droit xor coordinates in \x1,\y1 and use it for path
  \draw[thick] let \p1 = (bloc droit xor) in
  (bloc milieu BC.south) -- ++ (0,-0.5) -- ++(2.5,0) |- (2.5,\y1) -- 
  node[anchor = south] {$C_2$} ++ (0.5, 0)  node(dash start){};

  \draw[thick, <-] let \p1 = (bloc droit xor) in
  (bloc droit xor.west) -- node[anchor = south] {$C_{n-1}$}
  ++ (-0.6, 0) node (dash end) {};
  \draw[thick, dashed] (dash start.west) -- (dash end.east);


 %  |- node[anchor = south] {$C_2$}
 % (bloc droit xor.west) node[anchor = south east] {$C_{n-1}$};


\end{tikzpicture}
\end{document}