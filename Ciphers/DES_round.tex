%% Author: Aurélien Greuet (aureliengreuet@gmail.com)
%%
%% This work is licensed under the Creative Commons Attribution
%% Attribution-ShareAlike 4.0 International (CC BY-SA 4.0).
%% To view a copy of this license, visit
%% https://creativecommons.org/licenses/by-sa/4.0/

\documentclass{article}
\usepackage[utf8]{inputenc}
\usepackage{tikz}

\usetikzlibrary{backgrounds,calc,shapes,shapes.symbols, fit,
  decorations.pathmorphing, decorations.pathreplacing, positioning}
\usepackage[graphics,tightpage,active]{preview}
\PreviewEnvironment{tikzpicture}

\definecolor{Orange}{RGB}{233,93,45}
\definecolor{OrangeYellow}{RGB}{238,127,0}
\definecolor{Yellow}{RGB}{250,187,0}
\definecolor{Blue}{RGB}{0,81,158}
\definecolor{VioletBlue}{RGB}{87,35,129}
\definecolor{Red}{RGB}{226,0,43}
\definecolor{Violet}{RGB}{165,51,136}
\definecolor{Average}{RGB}{124,164,33}
\definecolor{Green}{RGB}{0,134,51}
\definecolor{Cyan}{RGB}{0,152,215}

\tikzset{table text/.style={matrix, column sep=-\pgflinewidth, nodes={rectangle}, minimum width=0.8em, minimum height=0.8em,
    inner sep = 0}}
\tikzset{table/.style={matrix, column sep=-\pgflinewidth, nodes={rectangle}, minimum width=3pt, minimum height=10pt,
    inner sep =0}}
\tikzset{table test/.style={matrix, column sep=-\pgflinewidth, nodes={rectangle}, inner sep =0}, minimum width = 3pt}
\tikzset{case/.style={draw, nodes={rectangle}, fill=Green!60,}}
\tikzset{case exp/.style={draw, nodes={rectangle}, fill=OrangeYellow!60,}}
\tikzset{case cipher/.style={draw, nodes={rectangle}, fill=Orange!60,}}
\tikzset{case key/.style={draw, nodes={rectangle}, fill=Red!60,}}
\tikzset{Sbox/.style={draw, nodes={rectangle}, fill=VioletBlue!60, text width=7pt, text centered}}
\tikzset{Perm/.style={draw, nodes={rectangle}, fill=Yellow!60, text centered}}
\tikzset{key/.style={draw, nodes={rectangle}, fill=Red!60, text centered}}
\tikzset{arrow/.style={inner sep=0pt}}
\tikzset{Sbox output/.style={inner sep=0pt}}
\tikzset{xor/.style={inner sep=0pt}}

\newcommand{\myarrow}{
  \begin{tikzpicture}
    \draw[->] (0,0.3) -- (0,0);
  \end{tikzpicture}
}

\newcommand{\SboxInit}[1]{
  \matrix[ampersand replacement=\&] (S#1 input)  [table]{
    \node[arrow] (S#1 input dummy) {\myarrow}; \& \node[arrow] {\myarrow}; \& \node[arrow] {\myarrow}; \& \node[arrow] {\myarrow}; \& \node[arrow] {\myarrow}; \&
    \node[arrow] {\myarrow};\\
  };

  \node[Sbox, below = 0pt of S#1 input] (S#1) {$\!S_#1$};

  \matrix (S#1 output)  [ampersand replacement=\&, table test, below = -0.2pt of S#1]{
    \node[Sbox output] (S#1 output 1) {}; \& \node[Sbox output] (S#1 output 2) {}; \& \node[Sbox output] (S#1 output 3) {}; \& \node[Sbox output] (S#1 output 4) {};\\
  };

  \matrix (S#1 output dummy)  [ampersand replacement=\&, table test, below = 10pt of S#1 output]{
    \node[Sbox output] (S#1 output dummy 1) {}; \& \node[Sbox output] (S#1 output dummy 2) {}; \& \node[Sbox output] (S#1 output
    dummy 3) {}; \& \node[Sbox output] (S#1 output dummy 4) {};\\
  };

  \draw[-] (S#1 output 1.north) -- (S#1 output dummy 1);
  \draw[-] (S#1 output 2.north) -- (S#1 output dummy 2);
  \draw[-] (S#1 output 3.north) -- (S#1 output dummy 3);
  \draw[-] (S#1 output 4.north) -- (S#1 output dummy 4);
}

\newcommand{\Sbox}[2]{
  \matrix[ampersand replacement=\&, right = 0.2 pt of S#2 input] (S#1 input)  [table]{
    \node[arrow] {\myarrow}; \& \node[arrow] {\myarrow}; \& \node[arrow] {\myarrow}; \& \node[arrow] {\myarrow}; \& \node[arrow] {\myarrow}; \&
    \node[arrow] (S#1 input dummy) {\myarrow};\\
  };

  \node[Sbox, below = 0pt of S#1 input] (S#1) {$\!S_#1$};

  \matrix (S#1 output)  [ampersand replacement=\&, table test, below = -0.2pt of S#1]{
    \node[Sbox output] (S#1 output 1) {}; \& \node[Sbox output] (S#1 output 2) {}; \& \node[Sbox output] (S#1 output 3) {}; \& \node[Sbox output] (S#1 output 4) {};\\
  };

  \matrix (S#1 output dummy)  [ampersand replacement=\&, table test, below = 10pt of S#1 output]{
    \node[Sbox output] (S#1 output dummy 1) {}; \& \node[Sbox output] (S#1 output dummy 2) {}; \& \node[Sbox output] (S#1 output
    dummy 3) {}; \& \node[Sbox output] (S#1 output dummy 4) {};\\
  };

  \draw[-] (S#1 output 1.north) -- (S#1 output dummy 1);
  \draw[-] (S#1 output 2.north) -- (S#1 output dummy 2);
  \draw[-] (S#1 output 3.north) -- (S#1 output dummy 3);
  \draw[-] (S#1 output 4.north) -- (S#1 output dummy 4);
}





\begin{document}
\sffamily
\begin{tikzpicture}[node distance = 1.5cm,>=latex]

\SboxInit{1}
\Sbox{2}{1}
\Sbox{3}{2}
\Sbox{4}{3}
\Sbox{5}{4}
\Sbox{6}{5}
\Sbox{7}{6}
\Sbox{8}{7}

%% xor
\draw[thick] ([yshift=-1pt,xshift=-0.2pt]S1 input dummy.north) -- ([yshift=-1pt,xshift=0.2pt]S8 input dummy.north);
\node[inner sep=0] (dummy middle input) at ($(S1 input dummy)!0.5!(S8 input dummy)$) {};
\node (xor) [xor, above = 30pt of dummy middle input] {\LARGE $\oplus$};
\draw[->, thick] (xor) -- ([yshift=4pt]dummy middle input.north);

\draw[thick] ([xshift=-0.2pt]S1 output dummy 1.north) -- ([xshift=0.2pt]S8 output dummy 4.north);


%% expanded block
  \matrix (exp block) [table text, above  = 20pt of xor, xshift=-35pt]{
    \node[case exp] {}; &  \node[case exp] {}; &  \node[case exp] {}; &  \node[case exp] {}; &  \node[case exp] {}; & \node[case exp] {}; &
    \node[case exp] {}; & \node[case exp] {};\\
  };

%% expansion
\node[Perm] (exp) [above = 20 pt of exp block] {Expansion};

%% half block
  \node[above of = exp] (half block label) {\footnotesize Half Block};
  \matrix (half block) [table text, below = 0.5pt of half block label]{
    \node[case] {}; &  \node[case] {}; &  \node[case] {}; &  \node[case] {}; &  \node[case] {}; & \node[case] {}; \\
  };

%% subkey
  \matrix (subkey) [table text, right = 13 pt of half block.east]{
    \node[case key] {}; &  \node[case key] {}; &  \node[case key] {}; &  \node[case key] {}; &  \node[case key] {}; & \node[case key] {}; &
    \node[case key] {}; & \node[case key] {};\\
  };
  \node[above =0.5pt of subkey] (subkeylabel) {\footnotesize Subkey};
  \draw[->, thick] (subkey.south) |- (xor.east);

%% permutation
\node[inner sep=0] (dummy middle output) at ($(S1 output dummy 1)!0.5!(S8 output dummy 4)$) {};
\node[Perm, below = 15 pt of dummy middle output] (perm) {Permutation};
\draw[->, thick] (dummy middle output.north) -- (perm);

\draw[->, thick] (half block.south) -- (exp.north);
\draw[->, thick] (exp.south) -- (exp block.north);
\draw[->, thick] (exp block.south) |- (xor.west);

\node[inner sep=0, below = 15 pt of perm] (dummy middle bottom) {};
\draw[->, thick] (perm) -- (dummy middle bottom);

\end{tikzpicture}
\end{document}