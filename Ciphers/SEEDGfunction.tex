%% Author: Aurélien Greuet (aureliengreuet@gmail.com)
%%
%% This work is licensed under the Creative Commons Attribution
%% Attribution-ShareAlike 4.0 International (CC BY-SA 4.0).
%% To view a copy of this license, visit
%% https://creativecommons.org/licenses/by-sa/4.0/

\documentclass{article}
\usepackage[utf8]{inputenc}
\usepackage{tikz}
\usepackage[graphics,tightpage,active]{preview}


\usetikzlibrary{backgrounds,calc,shapes,shapes.symbols, fit,
  decorations.pathmorphing, decorations.pathreplacing, positioning}
\PreviewEnvironment{tikzpicture}


\definecolor{Blue}{RGB}{0,81,158}
\definecolor{Red}{RGB}{226,0,43}
\definecolor{Yellow}{RGB}{250,187,0}
\definecolor{Orange}{RGB}{233,93,45}
\definecolor{Green}{RGB}{0,134,51}

\tikzset{table/.style={matrix, column sep=-\pgflinewidth, nodes={rectangle},
    minimum width=2.5em, minimum height=2em,
    inner sep = 0}}
\tikzset{case/.style={draw, nodes={rectangle}, fill=Yellow!40,}}
\tikzset{case red/.style={draw, nodes={rectangle}, fill=Red!40,}}
\tikzset{case blue/.style={draw, nodes={rectangle}, fill=Blue!40,}}
\tikzset{xor/.style={inner sep=0pt}}



\begin{document}
\sffamily
\begin{tikzpicture}[node distance = 3.5em, >=latex]

  \matrix (X)  [table]
  {
    \node[case] (X3) {$X_3$}; &  \node[case] (X2) {$X_2$}; &
    \node[case] (X1) {$X_1$}; & \node[case] (X0) {$X_0$}; \\
  };

  \matrix (S)  [table, below of = X]
  {
    \node[case red] (S21) {$S_2$}; &  \node[case red] (S11) {$S_1$}; &
    \node[case red] (S22) {$S_2$}; & \node[case red] (S12) {$S_1$}; \\
  };

  \draw[->, thick] (X0) -- (S12);
  \draw[->, thick] (X1) -- (S22);
  \draw[->, thick] (X2) -- (S11);
  \draw[->, thick] (X3) -- (S21);

\begin{scope}[node distance = 8 em]
  \node (phantom) [below of = S] {};
  \node (phantom2) [below of = phantom] {};
\end{scope}

  \matrix (bloc2)  [table, left = 0pt of phantom]
  {
    \node[case blue] (bloc2m3) {$\&m_3$}; &  \node[case blue] (bloc2m2) {$\&m_2$}; &
    \node[case blue] (bloc2m1) {$\&m_1$}; & \node[case blue] (bloc2m0) {$\&m_0$}; \\
  };

  \matrix (bloc1)  [table, left = 6pt of bloc2]
  {
    \node[case blue] (bloc1m3) {$\&m_3$}; &  \node[case blue] (bloc1m2) {$\&m_2$}; &
    \node[case blue] (bloc1m1) {$\&m_1$}; & \node[case blue] (bloc1m0) {$\&m_0$}; \\
  };

  \matrix (bloc3)  [table, right = 0pt of phantom]
  {
    \node[case blue] (bloc3m3) {$\&m_3$}; &  \node[case blue] (bloc3m2) {$\&m_2$}; &
    \node[case blue] (bloc3m1) {$\&m_1$}; & \node[case blue] (bloc3m0) {$\&m_0$}; \\
  };

  \matrix (bloc4)  [table, right = 6pt of bloc3]
  {
    \node[case blue] (bloc4m3) {$\&m_3$}; &  \node[case blue] (bloc4m2) {$\&m_2$}; &
    \node[case blue] (bloc4m1) {$\&m_1$}; & \node[case blue] (bloc4m0) {$\&m_0$}; \\
  };


  \foreach \y/\z in {11/2, 12/4, 21/1, 22/3}{
    \foreach \x in {3, ..., 0} {
      \draw[->, thick] (S\y.south) -- (bloc\z m\x.north);
    }
  }

  \node (xor3) [right of = phantom2, color = Red, xor] {\huge$\oplus$};
  \node (xor4) [right = of xor3, color = Blue, xor] {\huge$\oplus$};
  \node (xor2) [left of = phantom2, color = Orange, xor] {\huge$\oplus$};
  \node (xor1) [left = of xor2, color = Green, xor] {\huge$\oplus$};

  \foreach \x/\y in {1/2, 2/1, 3/0, 4/3} {
    \draw[->, thick, color = Green] (bloc\x m\y.south) -- (xor1);
  }
  \foreach \x/\y in {1/1, 2/0, 3/3, 4/2} {
    \draw[->, thick, color = Orange] (bloc\x m\y.south) -- (xor2);
  }
  \foreach \x/\y in {1/0, 2/3, 3/2, 4/1} {
    \draw[->, thick, color = Red] (bloc\x m\y.south) -- (xor3);
  }
  \foreach \x/\y in {1/3, 2/2, 3/1, 4/0} {
    \draw[->, thick, color = Blue] (bloc\x m\y.south) -- (xor4);
  }


  \matrix (Z)  [table, below = of phantom2]
  {
    \node[case] (Z3) {$Z_3$}; &  \node[case] (Z2) {$Z_2$}; &
    \node[case] (Z1) {$Z_1$}; & \node[case] (Z0) {$Z_0$}; \\
  };

  \foreach \x/\y in {1/3, 2/2, 3/1, 4/0} {
    \draw[->, thick] (xor\x.south) -- (Z\y.north);
  }

  \node[fit=(S)(bloc1)(bloc2)(bloc3)(bloc4)(xor1)(xor2)(xor3)(xor4),
  draw, dashed, thick, inner sep = 9pt] (G){};

\end{tikzpicture}
\end{document}