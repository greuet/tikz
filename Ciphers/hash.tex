%% Author: Aurélien Greuet (aureliengreuet@gmail.com)
%%
%% This work is licensed under the Creative Commons Attribution
%% Attribution-ShareAlike 4.0 International (CC BY-SA 4.0).
%% To view a copy of this license, visit
%% https://creativecommons.org/licenses/by-sa/4.0/

\documentclass{article}

\usepackage{tikz}
\usepackage[utf8]{inputenc}
\usepackage[graphics,tightpage,active]{preview}
\PreviewEnvironment{tikzpicture}
\usetikzlibrary{arrows,decorations.pathmorphing, decorations.pathreplacing,
  backgrounds,fit,positioning,shapes.symbols,chains}

\definecolor{VioletBlue}{RGB}{87,35,129} 

\begin{document}
\sffamily
\begin{tikzpicture}[->, >=latex, shorten >=1pt, auto, node
  distance=4cm, thick,
  key node/.style={font=\sffamily\bfseries\LARGE, text width=1cm, text centered}, 
  text node/.style={rectangle,draw, font=\scriptsize}, text width=6cm, text justified]

  \node[text node] (plaintext) { A hash function is any function that can be used to map digital data of arbitrary size to
    digital data of fixed size. The values returned by a hash function are called hash values, hash codes, hash sums, or
    simply hashes. One practical use is a data structure called a hash table, widely used in computer software for rapid
    data lookup. Hash functions accelerate table or database lookup by detecting duplicated records in a large file. An
    example is finding similar stretches in DNA sequences. They are also useful in cryptography. A cryptographic hash
    function allows one to easily verify that some input data matches a stored hash value, but makes it hard to construct
    any data that would hash to the same value or find any two unique data pieces that hash to the same value. This
    principle is used by the PGP algorithm for data validation and by many password checking systems.

    A cryptographic hash function is a hash function which is considered practically impossible to invert, that is, to
    recreate the input data from its hash value alone. These one-way hash functions have been called "the workhorses of
    modern cryptography". The input data is often called the message, and the hash value is often called the message digest
    or simply the digest.  };

  \node (ulcorner) [right = 40pt of plaintext, yshift=40pt] {};
  \node (dlcorner) [right = 40pt of plaintext, yshift=-40pt] {};
  \node (urcorner) [right = 75pt of plaintext, yshift=10pt] {};
  \node (drcorner) [right = 75pt of plaintext, yshift=-10pt] {};
  \draw[-, fill=VioletBlue!60] (ulcorner.west) -- (dlcorner.west) -- (drcorner.west) -- (urcorner.west) -- cycle;

  \node[key node] (H) [right = 40 pt of plaintext] {H};
  \node[text node, text width=4.25cm] (hashedtext) [right of=H] {45d9dcb79e8ea15a0458d27a8d1df1d5};


  \draw[->, ultra thick] (plaintext) -- (H);
  \draw[->, ultra thick] (H) -- (hashedtext);

\end{tikzpicture}
\end{document}